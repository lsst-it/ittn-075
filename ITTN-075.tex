\documentclass[PMO,authoryear,toc]{lsstdoc}
\input{meta}

% Package imports go here.

% Local commands go here.

%If you want glossaries
%\input{aglossary.tex}
%\makeglossaries

\title{Rubin IPsec Tunnels}

% This can write metadata into the PDF.
% Update keywords and author information as necessary.
\hypersetup{
    pdftitle={Rubin IPsec Tunnels},
    pdfauthor={Julio Constanzo},
    pdfkeywords={}
}

% Optional subtitle
% \setDocSubtitle{A subtitle}

\author{%
Julio Constanzo
}

\setDocRef{ITTN-075}
\setDocUpstreamLocation{\url{https://github.com/lsst-it/ittn-075}}

\date{\vcsDate}

% Optional: name of the document's curator
% \setDocCurator{The Curator of this Document}

\setDocAbstract{%
Rubin IPsec Tunnels 
}

% Change history defined here.
% Order: oldest first.
% Fields: VERSION, DATE, DESCRIPTION, OWNER NAME.
% See LPM-51 for version number policy.
\setDocChangeRecord{%
  \addtohist{1}{YYYY-MM-DD}{Unreleased.}{Julio Constanzo}
}


\begin{document}

% Create the title page.
\maketitle
% Frequently for a technote we do not want a title page  uncomment this to remove the title page and changelog.
% use \mkshorttitle to remove the extra pages

% ADD CONTENT HERE
% You can also use the \input command to include several content files.

\appendix
% Include all the relevant bib files.
% https://lsst-texmf.lsst.io/lsstdoc.html#bibliographies
\section{References} \label{sec:bib}
\renewcommand{\refname}{} % Suppress default Bibliography section
\bibliography{local,lsst,lsst-dm,refs_ads,refs,books}

% Make sure lsst-texmf/bin/generateAcronyms.py is in your path
\section{Acronyms} \label{sec:acronyms}
\addtocounter{table}{-1}
\begin{longtable}{p{0.145\textwidth}p{0.8\textwidth}}\hline
\textbf{Acronym} & \textbf{Description}  \\\hline

AES & Advanced Encryption Standard \\\hline
B & Byte (8 bit) \\\hline
CPU & Central Processing Unit \\\hline
DWDM & Dense Wave Division Multiplex \\\hline
EOS & Engineering Operations Services \\\hline
ESP & Early Science Program \\\hline
GRE & Generic Routing Encapsulation \\\hline
HQ & Head Quarters \\\hline
IP & Internet Protocol \\\hline
IPsec & Internet Protocol Security \\\hline
IT & Information Technology \\\hline
ITTN & IT Technote \\\hline
L2 & Lens 2 \\\hline
L3 & Lens 3 \\\hline
LAG & List of Acronyms and Glossary \\\hline
LFA & Large File Annex \\\hline
LHN & long haul network \\\hline
MAC & Media Access Control \\\hline
MLAG & Multi-chassis Link Aggregation \\\hline
MTU & Maximum Transmission Unit \\\hline
NAT & Network Address Translation \\\hline
OSPF & Open Short Path First \\\hline
PMO & Project Management Office \\\hline
QSFP & Quad Small Form Factor Plugable \\\hline
SA & System and Services Acquisition \\\hline
SLAC & SLAC National Accelerator Laboratory \\\hline
SPI & Schedule Performance Index \\\hline
SRCF & Stanford Research Computing Facility \\\hline
TAC & Time Allocation Committee \\\hline
TCP & Transmission Control Protocol \\\hline
US & United States \\\hline
VPN & virtual private network \\\hline
VRF & Virtual Routing and Forwarding \\\hline
WAN & Wide Area Network \\\hline
\end{longtable}

% If you want glossary uncomment below -- comment out the two lines above
%\printglossaries





\end{document}
